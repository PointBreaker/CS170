\documentclass{article}
\usepackage{cs170}
\begin{document}

\question{Study Group}
List the names and SIDs of the members in your study group.
If you have no collaborators, you must explicitly write ``none''

\question{A Reduction Warm-up}

\noindent In the Undirected Rudrata path problem (aka the Hamiltonian Path Problem), we are given a graph $G$ with undirected edges
as input and want to determine if there exists a path in $G$ that uses every vertex exactly once. \\

\noindent In the Longest Path in a DAG, we are given a DAG, and a variable $k$
as input and want to determine if there exists a path in the DAG that is of length $k$ or more. \\

\noindent Is the following reduction correct? Please justify your answer.\\

\noindent Undirected Rudrata Path can be reduced to Longest Path in a DAG. Given the undirected graph $G$, we will use DFS to find a traversal of $G$
and assign directions to all the edges in $G$ based on this traversal. In other words, the edges will point in the same direction they were traversed and back
edges will be omitted, giving us a DAG. If the longest path in this DAG has $|V| - 1$ edges then there must be a Rudrata path in $G$ since any simple
path with $|V| - 1$ edges must visit every vertex, so if this is true, we can say there exists a Rudrata path in the original graph.
Since running DFS takes polynomial time ($O(|V| + |E|)$), this reduction is valid.



\question{Decision vs. Search vs. Optimization}
	Recall that a vertex cover is a set of vertices in a graph such that every edge is adjacent to at least one vertex in this set.

	The following are three formulations of the \textsc{vertex cover} problem:
	\begin{itemize}
		\item As a \textit{decision problem}: Given a graph $G$, return TRUE if it has a vertex cover of size at most $b$, and FALSE otherwise.
		\item As a \textit{search problem}: Given a graph $G$, find a vertex cover of size at most $b$ (that is, return the actual vertices), or report that none exists.
		\item As an \textit{optimization problem}: Given a graph $G$, find a minimum vertex cover. \\
	\end{itemize}
	At first glance, it may seem that search should be harder than decision, and that optimization should be even harder. We will show that if any one can be solved in polynomial time, so can the others.
%	For the following parts, describe your algorithms precisely; justify correctness and the number of times that the black box is queried (asymptotically).

	\begin{subparts}
		\subpart Suppose you are handed a black box that solves \textsc{vertex cover (decision)} in polynomial time. Give an algorithm that solves \textsc{vertex cover (search)} in polynomial time.

		\subpart Similarly, suppose we know how to solve \textsc{vertex cover (search)} in polynomial time. Give an algorithm that solves \textsc{vertex cover (optimization)} in polynomial time.

	\end{subparts}




\question{Reduction to 3-Coloring}

Given a graph $G = (V, E)$, a valid 3-coloring assigns each vertex in the graph a color from $\{0, 1, 2\}$ such that for any edge $(u, v)$, $u$ and $v$ have different colors. In the 3-coloring problem, our goal is to find a valid 3-coloring if one exists. In this problem, we will give a reduction from 3-SAT to the 3-coloring problem. Since we know that 3-SAT is NP-Hard (there is a reduction to 3-SAT from every NP problem), this will show that 3-coloring is NP-Hard (there is a reduction to 3-coloring from every NP problem).

In our reduction, the graph will start with three special vertices, labelled ``True'', ``False'', and ``Base'', and the edges (True, False), (True, Base), and (False, Base).
\begin{subparts}

\subpart Consider the following graph, which we will call a ``gadget'':

\begin{figure}[h] \centering
% \graphicspath{{"src/problems/3-sat"}}
\includegraphics[scale=0.6]{problems/3-sat/gadg1.png}
\end{figure}

Show that in any valid 3-coloring of this graph which does not assign the color 2 to any of the gray vertices, the gray vertex on the right is assigned the color 1 only if one of the gray vertices on the left is assigned the color 1.

\subpart For each variable $x_i$ in a 3-SAT formula, we will create a
pair of vertices labeled $x_i$ and $\lnot x_i$. How should we add
edges to the graph such that in any valid 3-coloring, one of
$x_i, \lnot x_i$ is assigned the same color as True and the other is assigned the same color as False?
\subpart We observe the following about the graph we are creating in the reduction:
\begin{enumerate}[(i)]
\item For any vertex, if we have the edges ($v$, False) and ($v$, Base) in the graph, then in any valid 3-coloring $v$ will be assigned the same color as True.

\item Through brute force one can also show that in the gadget, for any assignment of colors to gray vertices such that:
\begin{enumerate}[(1)]
\item All gray vertices are assigned the color 0 or 1
\item The gray vertex on the right is assigned the color 1
\item At least one gray vertex on the left is assigned the color 1
\end{enumerate}
Then there is a valid coloring for the black vertices in the gadget.
\end{enumerate}

Using these observations and your answers to the previous parts, give
a reduction from 3-SAT to 3-coloring. Prove that your reduction is correct.

\end{subparts}
\end{document}