\documentclass{article}
\usepackage{cs170}
\begin{document}

\documentclass[11pt]{article}
\usepackage{cs170}


\def\title{Homework 5}
\def\duedate{3/1/2022, at 10:00 pm (grace period until 11:59pm)}


\begin{document}
\maketitle


Due \textbf{\duedate}


\question{Study Group}
List the names and SIDs of the members in your study group.
If you have no collaborators, you must explicitly write “none”.

\question{Copper Pipes}
Bubbles has a copper pipe of length n inches and an array of nonnegative integers that contains prices of all pieces of size smaller than $n$. He wants to find the maximum value he can make by cutting up the pipe and selling the pieces. For example, if length of the pipe is $8$ and the values of different pieces are given as following, then the maximum obtainable value is $22$ (by cutting in two pieces of lengths $2$ and $6$).
\newline

\begin{tabular}{ c | c  c  c  c  c  c  c  c }
length   & 1 &  2 & 3 & 4 & 5 & 6 & 7 & 8  \\ \hline
price    & 1 & 5 & 8 & 9 & 10 & 17 & 17 & 20
\end{tabular}
\newline

Give a dynamic programming algorithm so Bubbles can find the maximum obtainable value given any pipe length and set of prices.  Clearly describe your algorithm, prove its correctness and runtime.

\question{Egg Drop} 


You are given $k$ identical eggs and an $n$ story
building. You need to figure out the highest floor
$\ell \in \{0, 1, 2, \ldots n\}$ that you can drop an egg from without
breaking it. Each egg will never break when dropped from floor $\ell$ or lower, and always breaks if dropped from floor $\ell+1$ or higher. ($\ell = 0$ means the egg always breaks). Once an egg breaks, you cannot use it any more. However, if an egg does not break, you can reuse it.

Let $f(n, k)$ be the minimum number of egg drops that are needed to find $\ell$ (regardless of the value of $\ell$).

\begin{subparts}
\subpart Find $f(1,k)$, $f(0, k)$, $f(n,1)$, and $f(n,0)$.

\subpart Find a recurrence relation for $f(n,k)$. \emph{Hint: Whenever you drop an egg, call whichever of the egg breaking/not breaking leads to more drops the ``worst-case event''. Since we need to find $\ell$ regardless of its value, you should assume the worst-case event always happens.}
\end{subparts}

\end{document}

\end{document}