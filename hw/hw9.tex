\documentclass{article}
\usepackage{cs170}
\begin{document}

\question{How to Gamble With Little Regret} 
Suppose that you are gambling at a casino. Every day you play at a slot machine, and your goal is to minimize your losses. We
model this as the experts problem. Every day you must take the advice
of one of $n$ experts (i.e. play at a slot machine). At the end of each day $t$, if you take advice from expert $i$, the advice costs you some $c_i^t$ in $[0,1]$.  You want to minimize the regret $R$, defined as:
$$R=\frac{1}{T}\left(\sum^{T}_{t=1}c^t_{i(t)} - \min_{1\leq i\leq n}  \sum^{T}_{t=1}c_i^t\right)$$  
where $i(t)$ is the expert you choose on day $t$. Notice that in this definition, you are comparing your losses to the best expert, rather than the best
overall strategy.

Your strategy will
be probabilities where $p_i^t$ denotes the probability with which you
choose expert $i$ on day $t$.  Assume an all powerful adversary
(i.e. the casino) can look at your strategy ahead of time and decide
the costs associated with each expert on each day. Let $C$ denote
the set of costs for all experts and all days. Compute $\max_C (\mathbb{E}[R])$, or the maximum
possible (expected) regret that the adversary can guarantee, for each
of the following strategies.

\begin{subparts}
 
\subpart Any deterministic strategy, i.e. for each
$t$, there exists some $i$ such that $p_i^t=1$.

\subpart Always choose an expert according to some fixed probability
distribution at every time step. That is, fix some $p_1,\ldots, p_n$,
and for all $t$, set $p_i^t=p_i$.

What distribution minimizes the regret of this strategy? In other words, what is \newline $\text{argmin}_{p_1,\ldots,p_n}\max_{C} (\mathop{\mathbb{E}}[R])$?\\
\end{subparts}

This analysis should conclude that a good strategy for the problem must necessarily be randomized and adaptive.

\question{Weighted Rock-Paper-Scissors}

You and your friend used to play rock-paper-scissors, and have the loser pay the winner 1 dollar. However, you then learned in CS170 that the best strategy is to pick each move uniformly at random, which took all the fun out of the game.

Your friend, trying to make the game interesting again, suggests playing the following variant: If you win by beating rock with paper, you get 4 dollars from your opponent. If you win by beating scissors with rock, you get 2 dollars. If you win by beating paper with scissors, you get 1 dollar.

\begin{subparts}

\subpart Draw the payoff matrix for this game.


\subpart Write a linear program to find the optimal strategy.

\end{subparts}

\question{Domination} 


In this problem, we explore a concept called \textit{dominated strategies}. Consider a zero-sum game with the following payoff matrix for the row player:

\begin{tabular}{r r|c|c|c| }
		\multicolumn{1}{r}{}
		& \multicolumn{1}{r}{}
		&  \multicolumn{1}{r}{Column:}
		& \multicolumn{1}{r}{}
		& \multicolumn{1}{r}{}\\
		& & A & B & C \\
		\cline{2-5}
		& D & 1 & 2 & -3 \\
		\cline{2-5}
		Row: & E & 3 & 2 & -2 \\
		\cline{2-5}
		& F & -1 & -2 & 2 \\
		\cline{2-5}
\end{tabular} \\

\begin{subparts}

\subpart If the row player plays optimally, can you find the probability that they pick $D$ without directly solving for the optimal strategy? Justify your answer. 

(Hint: How do the payoffs for the row player picking D compare to their payoffs for picking E?) 

\subpart Given the answer to part a, if the both players play optimally, what is the probability that the column player picks $A$?

\subpart Given the answers to part a and b, what are both players' optimal strategies?

\end{subparts}

Note: All parts of this problem can be solved without using an LP solver or solving a system of linear equations.	

\question{(Optional) Minimum Infinity-Norm Cut}
In the \textsc{Minimum Infinity-Norm Cut} problem, you are given a connected undirected graph $G=(V, E)$ with positive edge weights $w_e$, and you are asked to find a cut in the graph where the largest edge in the cut is as small as possible (note that there is no notion of source or target; any cut with at least one node on each side is valid). 

Show how to reduce \textsc{Minimum Infinity-Norm Cut} to \textsc{Minimum Spanning Tree} in linear time. \textbf{Give a 3-part solution.}

\textit{Hint: Minimum Spanning Tree does not require edge weights to be positive.}
\end{document}